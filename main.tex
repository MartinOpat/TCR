\documentclass[
	a4paper,
	landscape,
	%twoside,
	10pt,
]{article}

% DO NOT REMOVE
\usepackage[utf8]{inputenc}
\usepackage{enumerate}
\usepackage{enumitem}
\usepackage{mathtools}
\usepackage{amsmath, amssymb}
\usepackage{xcolor}
\usepackage{float}
\usepackage{caption}
\usepackage{indentfirst}
\usepackage{changepage}
\usepackage{tabto}
\usepackage{lipsum}
\usepackage[export]{adjustbox}
\usepackage{multicol}
\usepackage{scalerel,stackengine}
\stackMath

\usepackage{xurl}
\usepackage{hyperref}
\hypersetup{
    colorlinks,
    citecolor = black,
    filecolor = black,
    linkcolor = black,
    urlcolor = violet
}

\usepackage{geometry}
\geometry{
    a4paper,
     % total = {170mm,257mm},
     % a3paper,
    left = 10mm,
    right = 10mm,
    top = 15mm,
    bottom = 15mm
 }

\usepackage{listings}
\lstset{
    language = C,
    tabsize = 2,
    showstringspaces = false,
    breaklines = true,
    basicstyle = \ttfamily,
    keywordstyle = \color[rgb]{0.1,0.3,0.7}\ttfamily,
    stringstyle = \color[rgb]{0.7,0.1,0.3}\ttfamily,
    commentstyle = \color[rgb]{0.3,0.4,0.3}\ttfamily,
    columns = fixed,
    numberstyle = \sffamily\scriptsize,
    backgroundcolor = \color[rgb]{0.95,0.95,0.95},
    frame = lines,
    framexleftmargin=5pt,
    numbers = left,
    numberstyle = \footnotesize
}

% Sets 1
\newcommand{\N}{\mathbb{N}}
\newcommand{\Z}{\mathbb{Z}}
\newcommand{\Q}{\mathbb{Q}}
\newcommand{\R}{\mathbb{R}}
\newcommand{\C}{\mathbb{C}}
\newcommand{\B}{\mathbb{B}}

% Sets 2
\newcommand{\Max}[1]{\mathsf{Max}\big\{#1\big\}~}
\newcommand{\Min}[1]{\mathsf{Min}\big\{#1\big\}~}
\newcommand{\no}[1]{\#\big\{#1\big\}}

% Puts text into a double box
\newcommand{\doubleboxed}{
    \begin{center}
        \fbox{\fbox{\parbox{5.9in}{\centering
        {\large
            \textbf{Data transmission: internet connections from server to the home (optical fiber, ADSL, cable) - 2 page summary} \\}}}
        }
    \end{center}
}

% Vectors
\newcommand{\ivec}{\boldsymbol{~\hat\imath}}
\newcommand{\jvec}{\boldsymbol{~\hat\jmath}}
\newcommand{\kvec}{\boldsymbol{~\hat k}}
\newcommand{\vecSpan}[1]{\text{Span} \left\{ #1 \right\}}
\newcommand{\mvect}[1]{\ol{#1}}
\newcommand{\vect}[1]{$\mvect{#1}$}
\newcommand{\rwhat}[1]{%
\savestack{\tmpbox}{\stretchto{%
  \scaleto{%
    \scalerel*[\widthof{\ensuremath{#1}}]{\kern.1pt\mathchar"0362\kern.1pt}%
    {\rule{0ex}{\textheight}}%WIDTH-LIMITED CIRCUMFLEX
  }{\textheight}% 
}{2.4ex}}%
\stackon[-6.9pt]{#1}{\tmpbox}%
}

% Matrices
\newcommand{\adj}[1][]{\text{adj}~#1}
\newcommand{\rank}[1][]{\text{rank}~#1}
\newcommand{\dimension}[1][]{\text{dim}~#1}
\newcommand{\nullspace}[1]{\mathcal{N} \left( #1 \right)}

% Renaming inbuild
\newcommand{\ol}[1]{\overline{#1}}

% Other
\newcommand{\abs}[1]{\left|#1\right|}


%*************** Section entries ***************
% \entry{name}{description}{snippet location}{complexity}{dependencies}
\newcommand{\entry}[3]{
	\subsubsection{#1}
	#2
	\ifthenelse{\equal{#3}{}}{}{
    % \immediate\write18{./process_c++_files.sh "#3"}
    \lstinputlisting{#3}
  }
}
\newcommand{\pyentry}[3]{
	\subsubsection{#1}
	#2
	\ifthenelse{\equal{#3}{}}{}{\lstinputlisting[language=python]{#3}}
}
\newcommand{\otherentry}[3]{
	\subsection{#1}
	#2
	\lstinputlisting[language=]{#3}
}

% \pagestyle{headandfoot}
% \runningheadrule
% \firstpageheader{\mycoursename}
%               {\mytitle~ Author: \myauthor}
%               {Page \thepage\ of \numpages}
% \runningheader{~}
%               {~}
%               {Page \thepage\ of \numpages}
% \firstpagefooter{}{}{}
% \runningfooter{}{}{}


\newenvironment{indented}[1][0.5cm] 
{\begin{adjustwidth}{#1}{0em}}
{\end{adjustwidth}}


\newenvironment{references}
{\begin{enumerate}[label={[\arabic*]}]}
{\end{enumerate}}

% Lin. Alg.

\newenvironment{amatrix}[2]{%
  \left[\begin{array}{@{}*{#1}{c}|*{#2}c@{}}
}{%
  \end{array}\right]
}

% Fill out
\newcommand{\mytitle}{~}
\newcommand{\myauthor}{Martin Opat}
\newcommand{\mycoursename}{TCR}


%*************** Table of Contents ***************
\usepackage[toc]{multitoc}			% multicolumn toc
\usepackage{tocloft}				% to reduce toc spacing
\renewcommand*{\multicolumntoc}{3}
% reduce section spacing in toc
\setlength{\cftbeforesecskip}{-1pt}
\setlength{\cftbeforesubsecskip}{-1.5pt}
% remove the toc title
\makeatletter
\renewcommand{\@cftmaketoctitle}{}
\makeatother

\begin{document}
\setcounter{tocdepth}{3}
\tableofcontents

\begin{multicols}{3}
\section{Setup}
\subsubsection{Tips}
\textbf{Test session}: Check \texttt{\_\_int128}, GNU builtins, and end of line whitespace requirements. \\
\textbf{C++ var. limits}: 
\texttt{int} $-2^{31},~2^{31}-1$ \\
\texttt{ll} $-2^{63},~2^{63}-1$ \\
\texttt{ull} $0,~2^{64}-1$ \\
\texttt{\_\_int128} $-2^{127},~2^{127}-1$\\
\texttt{ld} $-1.7e308,~1.7e308$, $18$ digits precision
\subsubsection{Xmodmap setup}
\texttt{remove Lock = Caps\_Lock} \\
\texttt{keysym Escape = Caps\_Lock} \\
\texttt{keysym Caps\_Lock = Escape} \\
\texttt{add Lock = Caps\_Lock} \\

\entry{header.h}{~}{code/header.h}
% \otherentry{Bash for c++ compile with header.h}{~}{compile_c++.sh}
% \otherentry{Bash for run tests c++}{~}{code/test_cpp.sh}
% \otherentry{Bash for run tests python}{~}{code/test_py.sh}
\entry{Aux. helper C++}{~}{code/auxiliary.cpp}
\pyentry{Aux. helper python}{~}{code/auxiliary.py}

\section{Python}
\subsection{Graphs}
	\pyentry{BFS}{~}{code/Graphs/bfs.py}
	\pyentry{Dijkstra}{~}{code/Graphs/dijkstra_filtered.py}
	\pyentry{Topological Sort}{topological sorting of a DAG}{code/Graphs/topological_sort.py}
	\pyentry{Kruskal (UnionFind)}{Min. span. tree}{code/Graphs/kruskal_new.py}
	\pyentry{Prim}{Min. span. tree - good for dense graphs}{code/Graphs/prim.py}
	% TODO: If space add longest path
	
\subsection{Num. Th. / Comb.}
	\pyentry{nCk \% prime}{p must be prime and k $<$ p}{code/Number Theory/nCk_mod_prime.py}
	\pyentry{Sieve of E.}{$O(n)$ so actually faster than C++ version, but more memory}{code/Number Theory/better_sieve.py}
	\pyentry{Modular Inverse}{of a mod b}{code/Number Theory/modinv.py}
	\pyentry{Chinese rem.}{an x such that $\forall$ y,m: yx = 1 mod m requires all m,m' to be $>=$1 and coprime}{code/Number Theory/chinese_remainder.py}
	\pyentry{Bezout}{~}{code/Number Theory/bezout.py}
	\pyentry{Gen. chinese rem.}{~}{code/Number Theory/general_chinese_remainder.py}
	
\subsection{Strings}
	\pyentry{Longest common substr.}{(Consecutive) $O(mn)$ time, $O(m)$ space}{code/Strings/leastcommonsubstring.py}
	\pyentry{Longest common subseq.}{(Non-consecutive)}{code/Strings/lcs.py}
	\pyentry{KMP}{Return all matching pos. of P in T}{code/Strings/kmp.py}
	\pyentry{Suffix Array}{~}{code/Strings/suffixarray.py}
	\pyentry{Longest common pref.}{with the suffix array built we can do, e.g., longest common prefix of x, y with suffixarray where x,y are suffixes of the string used $O(\log n)$}{code/Strings/lcpref.py}
	\pyentry{Edit distance}{~}{code/Strings/edit_distance.py}
	\pyentry{Bitstring}{Slower than a set for many elements, but hashable. Also see \texttt{Hashing}}{code/Strings/bitstring.py}

\subsection{Geometry}
	\pyentry{Convex Hull}{~}{code/Geometry/convexhull.py}
	\pyentry{Geometry}{~}{code/Geometry/geometry.py}

\subsection{Other Algorithms}
	\pyentry{Rotate matrix}{~}{code/Other/rotate_matrix.py}

\subsection{Other Data Structures}
	% \pyentry{Segment Tree}{~}{code/Data Structures/segment_tree.py}
	\pyentry{Trie}{~}{code/Data Structures/trie.py}
	% \pyentry{RedBlack tree}{~}{code/Data Structures/llrbst.py}  % TODO: too long so just use cpp



\section{C++}
\subsection{Graphs}
    \entry{BFS}{~}{code/Graphs/bfs_filtered.cpp}
	\entry{DFS}{Cycle detection / removal}{code/Graphs/dfs.cpp}
	\entry{Dijkstra}{~}{code/Graphs/dijkstra_filtered.cpp}
	\entry{Floyd-Warshall}{~}{code/Graphs/warshall.cpp}
	\entry{Kruskal}{Minimum spanning tree of undirected weighted graph. $O(E \log E)$}{code/Graphs/kruskal.cpp}
	\entry{Hungarian algorithm}{Given J jobs and W workers (J $<=$ W), computes the minimum cost to assign each prefix of jobs to distinct workers.}{code/Graphs/hungarian.cpp}
	\entry{Suc. shortest path}{Calculates max flow, min cost}{code/Graphs/succ_shortest_path.cpp}
	\entry{Bipartite check}{~}{code/Graphs/bipartite_check.cpp}
    \entry{Bipartite matching (Hopcroft-Karp)}{
        Fast bipartite matching algorithm. Graph $g$ should be a list
        of neighbors of the left partition, and $btoa$ should be a vector full of
        -1's of the same size as the right partition. Returns the size of
        the matching. $btoa[i]$ will be the match for vertex $i$ on the right side,
        or $-1$ if it's not matched. Time: $O(\sqrt{V}E)$
    }{code/Graphs/hopcroftkarp.cpp}
	\entry{Find cycle directed}{~}{code/Graphs/find_cycle_directed.cpp}
	\entry{Find cycle undirected}{~}{code/Graphs/find_cycle_undirected.cpp}
	\entry{Tarjan's SCC}{~}{code/Graphs/tarjan.cpp}
	\entry{SCC edges}{Prints out the missing edges to make the input digraph strongly connected}{code/Graphs/scc_edges.cpp}
	% \entry{Find Bridges}{~}{code/Graphs/find_bridges.cpp}
	% \entry{Articulation points}{(i.e. cut off points)}{code/Graphs/find_articulation_points.cpp}
	\entry{Topological sort}{~}{code/Graphs/topo_sort.cpp}
	\entry{Bellmann-Ford}{Same as Dijkstra but allows neg. edges}{code/Graphs/bellmannford.cpp}
	\entry{Ford-Fulkerson}{Basic Max. flow}{code/Graphs/ford-fulk.cpp}
	\entry{Dinic max flow}{$O(V^2E)$, $O(Ef)$}{code/Graphs/dinic.cpp}

	% TODO: Re-implement this to 100% match our graph implementation (?)
	\entry{Edmonds-Karp}{(Max) flow algorithm with time $O(VE^2)$. To get edge flow values, compare capacities before and after, and take the positive values only.}{code/Graphs/EdmondsKarp.h}

\subsection{Dynamic Programming}
	\entry{Longest Incr. Subseq.}{~}{code/Dynamic Programming/lis.cpp}
	\entry{0-1 Knapsack}{Given a number of coins, calculate all possible distinct sums}{code/Dynamic Programming/0-1_knapsack.cpp}
	\entry{Coin change}{Total distinct ways to make sum using $n$ coins of different vals}{code/Dynamic Programming/coin_change.cpp}
	\entry{Longest common subseq.}{Optimization for each unique element appearing k-times}{code/Dynamic Programming/lcs_fast.cpp}

% \subsection{Trees}
	% \entry{Tree diameter}{~}{code/Graphs/tree_diameter.cpp}
	% \entry{Tree Node Count}{~}{code/Graphs/tree_node_count.cpp}

\subsection{Numerical}
	\entry{Template (for this section)}{~}{code/Numerical/template.cpp}
	  \entry{Polynomial}{~}{code/Numerical/Polynomial.h}
	  \entry{Poly Roots}{Finds the real roots to a polynomial.$O(n^2 \log(1/\epsilon))$}{code/Numerical/PolyRoots.h}
	
	  \entry{Golden Section Search}{Finds the argument minimizing the function $f$ in the interval $[a,b]$ assuming $f$ is unimodal on the interval, i.e. has only one local minimum and no local maximum. The maximum error in the result is $eps$. Works equally well for maximization with a small change in the code. See TernarySearch.h in the Various chapter for a discrete version. $O(\log((b-a) / \epsilon))$}{code/Numerical/GoldenSectionSearch.h}
	  \entry{Hill Climbing}{Poor man's optimization for unimodal functions.}{code/Numerical/HillClimbing.h}
	  \entry{Integration}{Simple integration of a function over an interval using Simpson's rule. The error should be proportional to $h^4$, although in practice you will want to verify that the result is stable to desired precision when epsilon changes.}{code/Numerical/Integrate.h}
	  \entry{Integration Adaptive}{Fast integration using an adaptive Simpson's rule.}{code/Numerical/IntegrateAdaptive.h}

	  % TODO: Add if space
	% \section{Matrices}
	%   \entry{Determinant.h}
	%   \entry{IntDeterminant.h}
	%   \entry{SolveLinear.h}
	%   \entry{SolveLinear2.h}
	%   \entry{SolveLinearBinary.h}
	%   \entry{MatrixInverse.h}
	  % \entry{MatrixInverse-mod.h}
	%   \entry{Tridiagonal.h}

		% TODO: Maybe consider switching
	% \section{Combinatorics}
	% \section{Fourier transforms}
		% \entry{FastFourierTransform.h}
		% \entry{FastFourierTransformMod.h}
		% \entry{NumberTheoreticTransform.h}
		% \entry{FastSubsetTransform.h}
	

\subsection{Num. Th. / Comb.}
	\entry{Basic stuff}{~}{code/Number Theory/elementary.cpp}
	\entry{Mod. exponentiation}{Or use pow() in python}{code/Number Theory/mod_exp.cpp}
	\entry{GCD}{Or math.gcd in python, std::gcd in C++}{code/Number Theory/gcd.cpp}
	\entry{Sieve of Eratosthenes}{~}{code/Number Theory/sieve.cpp}
	\entry{Fibonacci \% prime}{Starting $1,1,2,3,\hdots$}{code/Number Theory/fib_mod_prime.cpp}
	\entry{nCk \% prime}{~}{code/Number Theory/nCk_mod_prime.cpp}
	% \entry{Chin. rem. th.}{~}{code/Number Theory/chinese_rt.cpp}

\subsection{Strings}
	\entry{Z alg.}{KMP alternative (same complexities)}{code/Strings/z.cpp}
	\entry{KMP}{~}{code/Strings/kmp.cpp}
	\entry{Aho-Corasick}{Also can be used as Knuth-Morris-Pratt algorithm}{code/Strings/aho_corasick.cpp}
	\entry{Long. palin. subs}{Manacher - $O(n)$}{code/Strings/manacher.cpp}
	% \entry{Bitstring}{Slower than an unordered set (for many elements), but hashable. Also see \texttt{Hashing}}{code/Strings/bitstring.cpp}

\subsection{Geometry}
	\entry{essentials.cpp}{~}{code/Geometry/essentials.cpp}
	\entry{Two segs. itersec.}{~}{code/Geometry/intersec.cpp}
	\entry{Convex Hull}{~}{code/Geometry/convex_hull.cpp}

\subsection{Other Algorithms}
	\entry{2-sat}{~}{code/Combinatorics/2-sat.cpp}
	% \entry{Matrix Solve}{~}{code/Other/matrix_solve.cpp}  % Add if space (but probably not)
	% \entry{Matrix Exp.}{~}{code/Other/matrix_exp.cpp}
	\entry{Finite field}{For FFT}{code/Other/fin_field.cpp}
	\entry{Complex field}{For FFR}{code/Other/complex_field.cpp}
	\entry{FFT}{~}{code/Other/fft.cpp}
	\entry{Polyn. inv. div.}{~}{code/Other/poly.cpp}
	\entry{Linear recurs.}{Given a linear recurrence of the form
	$$a_n = \sum_{i=0}^{k-1} c_i a_{n-i-1}$$
	this code computes $a_n$ in $O(k \log k \log n)$ time.}{code/Other/lin_recursion.cpp}
	\entry{Convolution}{Precise up to 9e15}{code/Other/convo.cpp}
	\entry{Partitions of $n$}{Finds all possible partitions of a number}{code/Other/all_partitions.cpp}
	\entry{Ternary search}{Find the smallest i in $[a,b]$ that maximizes $f(i)$, assuming that $f(a) < \dots < f(i) \ge \dots \ge f(b)$. To reverse which of the sides allows non-strict inequalities, change the $<$ marked with (A) to $<=$, and reverse the loop at (B). To minimize $f$, change it to $>$, also at (B). $O(\log(b-a))$}{code/Other/TernarySearch.h}
	\entry{Hashing}{Also see \texttt{Primes} in \texttt{Other Mathematics}}{code/Other/hashing.cpp}

\subsection{Other Data Structures}
	\entry{Disjoint set}{(i.e. union-find)}{code/Data Structures/disjoint_set.cpp}
	\entry{Fenwick tree}{(i.e. BIT) eff. update + prefix sum calc. Can be generalized to arbitrary dimensions by duplicating loops.}{code/Data Structures/fenwick_tree.cpp}
	% \entry{Fenwick2d tree}{~}{code/Data Structures/fenwick2d.cpp}
	\entry{Trie}{~}{code/Data Structures/trie.cpp}
	\entry{Treap}{A binary tree whose nodes contain two values, a key and a priority, such that the key keeps the BST property}{code/Data Structures/treap.cpp}
	\entry{Segment tree}{~}{code/Data Structures/seg_tree.cpp}
	\entry{Lazy segment tree}{Uptimizes range updates}{code/Data Structures/lazy_seg_tree.cpp}
	\entry{Dynamic segment tree}{Sparse, i.e., larges values, i.e., not storred as an array}{code/Data Structures/segmenttree_dynamic.cpp}
	\entry{Suffix tree}{~}{code/Data Structures/suffix_tree.cpp}
	\entry{UnionFind}{~}{code/Data Structures/union_find.cpp}
	\entry{Indexed set}{Similar to set, but allows accessing elements by index using \texttt{find\_by\_order()} in $O(\log n)$}{code/Data Structures/indexed_set.cpp}
	\entry{Order Statistics Tree}{A set (not multiset!) with support for finding the n'th element, and finding the index of an element. To get a map, change \texttt{null\_type}.$O(\log N)$}{code/Data Structures/order_statistics_tree.cpp}
    \entry{Range minimum queries}{Answers range minimum queries in constant time after $O(V\log V)$ preproc.}{code/Data Structures/rmq.cpp}
	
\section{Other Mathematics}
	\subsection{Helpful functions}
		\entry{Euler's Totient Fucntion}
		{
			$n = p_1^{k_1-1}\cdot(p_1-1) \cdot \hdots \cdot p_r^{k_r-1}\cdot(p_r-1)$, where $p_1^{k_1} \cdot \hdots \cdot p_r^{k_r}$ is the prime factorization of $n$.
		}
		{code/Other Math/euler_totient_func.cpp}
		\pyentry{Totient (again but .py)}{~}{code/Other Math/totient.py} % TODO: Possibly remove one of the totient
		\subsubsection*{Formulas}
			$\Phi(n)$ counts all numbers in ${1,\hdots,n-1}$ coprime to $n$. \\
			$a^{\varphi (n)} \equiv 1 \mod n$, $a$ and $n$ are coprimes. \\
			$\forall e > \log_2 m:~ n^e \mod m = n^{\Phi(m) + e \mod \Phi(m)} \mod m$. \\
			$\text{gcd}(m,n) = 1 \Rightarrow \Phi(m\cdot n) = \Phi(m) \cdot \Phi(n)$. \\

		\entry{Pascal's trinagle}{$\binom{n}{k}$ is $k$-th element in the $n$-th row, indexing both from 0}{code/Other Math/pascal_triangle.cpp}

	\subsection{Theorems and definitions}
		% \subsubsection**{Fermat's little theorem}
		% $$a^p \equiv a \mod p$$
		\subsubsection*{Subfactorial (Derangements)}
		Permutations of a set such that none of the elements appear in
		their original position:
		$$!n = n! \sum_{i=0}^{n} \frac{(-1)^i}{i!}$$
		$$!(0) = 1, ~ !n = n \cdot !(n-1) + (-1)^n$$
		\begin{gather}
			!n = (n-1)(!(n-1)+!(n-2)) = \left[\frac{n!}{e}\right] \\
			!n = 1-e^{-1},~n\rightarrow\infty
		\end{gather}

		\subsubsection*{Binomials and other partitionings}
		$$\binom{n}{k} = \binom{n-1}{k}+\binom{n-1}{k-1} =
			\prod_{i=1}^k \frac{n-i+1}{i}$$ This last product may be computed
		incrementally since any product of $k'$ consecutive values is divisibleby
		$k'!$. \\
		Basic identities: The hockeystick identity: \\
		 $$\sum_{k=r}^n \binom{k}{r}
			= \binom{n+1}{r+1}$$
		or $$\sum_{k\leq n}\binom{r+k}{k} = \binom{r+n+1}{n}$$
		Also $$\sum_{k=0}^n \binom{k}{m} = \binom{n+1}{m+1}$$
		$$\sum_{i=0}^{n} \binom{n}{i} = 2^n$$

		For $n, m \geq 0$ and $p$ prime: write $n, m$ in base $p$, i.e.
		$n = n_k p^k + \dots + n_1 p + n_0$ and $m = m_k p^k + \dots m_1 p + m_0$. Then
		by Lucas theorem we have $\binom{n}{m} \equiv \prod_{i=0}^k \binom{n_i}{m_i}
		\mod p$, with the convention that $n_i < m_i \implies \binom{n_i}{m_i} =0$.

		\subsubsection*{Fibonacci} (See also number theory section) \\
 		$$\sum_{0 \leq k \leq n} \binom{n-k}{k} = F_{n+1}$$ 
		$$F_n = \frac{1}{\sqrt{5}} \left(\frac{1+\sqrt{5}}{2}\right)^n - \frac{1}{\sqrt{5}}\left(\frac{1-\sqrt{5}}{2}\right)^n$$
		$\sum_{i=1}^{n} {F_i} = F_{n+2}-1$, $\sum_{i=1}^{n} {F_i^2} = F_{n}F_{n+1}$
		$$\text{gcd}(F_m, F_n) = F_{\text{gcd}(m, n)}$$
		$$\text{gcd}(F_n, F_{n+1}) = \text{gcd}(F_n, F_{n+2}) = 1$$

		\subsubsection*{Bit stuff} $a + b = a \oplus b + 2(a\&b) = a | b + a\&b$. \\
		kth bit is set in $x$ iff $x \mod 2^{k-1} \geq 2^k$, or iff $x \mod 2^{k-1}  - x \mod 2^k \neq 0~(\text{i.e.} = 2^k)$ It comes handy when you need to look at the bits of the numbers which are pair sums or subset sums etc. \\
		$n \mod 2^i = n\&(2^i-1)$. \\
		$\forall k:~ 1 \oplus 2 \oplus \hdots \oplus (4k-1) = 0$


		\subsection{Geometry Formulas}
		\begin{align*}
			\textrm{Euler:}&&  1 + CC &= V - E + F\\
            \textrm{Pick:}&& \textrm{Area} &= {\small \textrm{itr pts}
            + \frac{\textrm{bdry pts}}2 - 1}
		\end{align*}
		Given a non-self-intersecting closed polygon on $n$ vertices, given as $(x_i, y_i)$, its centroid $(C_x, C_y)$ is given as:
		\begin{align*}
			C_x &= \frac{1}{6A} \sum_{i = 0}^{n - 1} (x_i + x_{i+1}) (x_i y_{i+1} - x_{i+1} y_i), \\
			C_y &= \frac{1}{6A} \sum_{i = 0}^{n - 1} (y_i + y_{i+1}) (x_i y_{i+1} - x_{i+1} y_i)
		\end{align*}
		\begin{equation*}
			A = \frac{1}{2} \sum_{i = 0}^{n - 1} (x_i y_{i+1} - x_{i+1} y_i) = \textrm{polygon area}
		\end{equation*}

		\subsubsection*{Inclusion-Exclusion}
		For appropriate $f$ compute $\sum_{S\subseteq T} (-1)^{|T\setminus S|} f(S)$,
		or if only the size of $S$ matters, $\sum_{s=0}^n (-1)^{n-s} \binom{n}{s}f(s)$.
		In some contexts we might use Stirling numbers, not binomial coefficients!

		Some useful applications:
		\begin{enumerate}
			\item[] \textbf{Graph coloring} Let $I(S)$ count the number
				of independent sets
				contained in $S \subseteq V$ ($I(\emptyset) = 1$,
				$I(S) = I(S\setminus v) + I(S\setminus N(v))$). Let
				$c_k = \sum_{S\subseteq V} (-1)^{|V\setminus S|} I(S)$. Then $V$
				is $k$-colorable iff $v > 0$. Thus we can compute the chromatic
				number of a graph in $O^*(2^n)$ time.
		\end{enumerate}

		\subsubsection*{Burnside's lemma}
			Given a group $G$ acting on a set $X$, the number of elements in $X$ up to
			symmetry is $$\frac{1}{|G|}\sum_{g\in G} |X^g|$$ with $X^g$ the elements of
			$X$ invariant under $g$. For example, if $f(n)$ counts ``configurations''
			of some sort of length $n$, and we want to count them up to rotational symmetry
			using $G = \mathbb{Z}/n\mathbb{Z}$, then

			$$g(n) = \frac{1}{n} \sum_{k=0}^{n-1} f(\gcd(n, k))
				= \frac{1}{n}\sum_{k \| n} f(k) \phi(n / k)$$

			I.e. for coloring with $c$ colors we have $f(k) = k^c$.

			Relatedly, in P\'olya's enumeration theorem we imagine $X$ as a set of $n$
			beads with $G$ permuting the beads (e.g. a necklace, with $G$ all rotations and
			reflections of the $n$-cycle, i.e. the dihedral group $D_n$).
			Suppose further that we had $Y$ colors, then
			the number of $G$-invariant colorings $Y^X / G$ is counted by

			$$\frac{1}{|G|}\sum_{g\in G} |Y|^{c(g)}$$

			with $c(g)$ counting the number of cycles of $g$ when viewed as a permutation
			of $X$. We can generalize this to a weighted version: if the color $i$ can
			occur exactly $r_i$ times, then this is counted by the coefficient of
			$t_1^{r_1}\dots t_n^{r_n}$ in the polynomial
			$$Z(t_1,\dots,t_n) = \frac{1}{|G|}\sum_{g\in G} \prod_{m\geq 1}
				(t_1^m+\dots+t_n^m)^{c_m(g)}$$
			where $c_m(g)$ counts the number of length $m$ cycles in $g$ acting as a
			permutation on $X$. Note we get the original formula by setting all $t_i=1$.
			Here $Z$ is the cycle index. Note: you can cleverly deal with even/odd sizes
			by setting some $t_i$ to $-1$.
		
	\subsubsection*{Lucas Theorem}
		If $p$ is prime, then:
		$$ \frac{p^a}{k} \equiv 0 (\mod p) $$
		Thus for non-negative integers $m = m_kp^k + \hdots + m_1 p + m_0$ and $n = n_k p^k + \hdots + n_1p + n_0$:
		$$ \frac{m}{n} =  \Pi_{i=0}^{k} \frac{m_i}{n_i} \mod p $$ Note: The fraction's mean integer division.

    \subsection{Recurrences}
        If $a_n = c_1 a_{n-1} + \dots + c_k a_{n-k}$, and $r_1, \dots, r_k$ are distinct roots of $x^k - c_1 x^{k-1} - \dots - c_k$, there are $d_1, \dots, d_k$ s.t.
        \[a_n = d_1r_1^n + \dots + d_kr_k^n. \]
        Non-distinct roots $r$ become polynomial factors, e.g. $a_n = (d_1n + d_2)r^n$.

	\subsection{Sums} \vspace{-1cm}
		\begin{align*}
			1^3 + 2^3 + 3^3 + \dots + n^3 &= \frac{n^2(n+1)^2}{4} \\
			1^4 + 2^4 + 3^4 + \dots + n^4 &= \frac{n(n+1)(2n+1)(3n^2 + 3n - 1)}{30}
		\end{align*}
	\subsection{Series} \vspace{-1cm}
		$$e^x = 1+x+\frac{x^2}{2!}+\frac{x^3}{3!}+\dots,\,(-\infty<x<\infty)$$
		$$\ln(1+x) = x-\frac{x^2}{2}+\frac{x^3}{3}-\frac{x^4}{4}+\dots,\,(-1<x\leq1)$$
		$$\sqrt{1+x} = 1+\frac{x}{2}-\frac{x^2}{8}+\frac{2x^3}{32}-\frac{5x^4}{128}+\dots,\,(-1\leq x\leq1)$$
		$$\sin x = x-\frac{x^3}{3!}+\frac{x^5}{5!}-\frac{x^7}{7!}+\dots,\,(-\infty<x<\infty)$$
		$$\cos x = 1-\frac{x^2}{2!}+\frac{x^4}{4!}-\frac{x^6}{6!}+\dots,\,(-\infty<x<\infty)$$

	\subsection{Quadrilaterals}
        With side lengths $a,b,c,d$, diagonals $e, f$, diagonals angle $\theta$, area $A$ and
        magic flux $F=b^2+d^2-a^2-c^2$:
        
        \[ 4A = 2ef \cdot \sin\theta = F\tan\theta = \sqrt{4e^2f^2-F^2} \]
        
         For cyclic quadrilaterals the sum of opposite angles is $180^\circ$,
        $ef = ac + bd$, and $A = \sqrt{(p-a)(p-b)(p-c)(p-d)}$.
        
        
	\subsection{Triangles}
        Side lengths: $a,b,c$\\
        Semiperimeter: $p=\dfrac{a+b+c}{2}$\\
        Area: % $A=\sqrt{p(p-a)(p-b)(p-c)}$\\
		\begin{gather*}
			[ABC]
			= rp
			= \frac 12 ab\sin\gamma \\
			= \frac{abc}{4R}
			= \sqrt{p(p-a)(p-b)(p-c)}
			= \frac 12\left| (B-A, C-A)^T \right|
		\end{gather*}
        Circumradius: $R=\dfrac{abc}{4A}$,~
        Inradius: $r=\dfrac{A}{p}$\\
        Length of median (divides triangle into two equal-area triangles): $m_a=\tfrac{1}{2}\sqrt{2b^2+2c^2-a^2}$\\
        Length of bisector (divides angles in two): $s_a=\sqrt{bc\left[1-\left(\dfrac{a}{b+c}\right)^2\right]}$\\
        Law of tangents: $\dfrac{a+b}{a-b}=\dfrac{\tan\dfrac{\alpha+\beta}{2}}{\tan\dfrac{\alpha-\beta}{2}}$

	\subsection{Trigonometry} \vspace{-1cm}
        \begin{align*}
        \tan(v+w)&{}=\dfrac{\tan v+\tan w}{1-\tan v\tan w}\\
        \sin v+\sin w&{}=2\sin\dfrac{v+w}{2}\cos\dfrac{v-w}{2}\\
        \cos v+\cos w&{}=2\cos\dfrac{v+w}{2}\cos\dfrac{v-w}{2}
        \end{align*}
        \[ (V+W)\tan(v-w)/2{}=(V-W)\tan(v+w)/2 \]
        where $V, W$ are lengths of sides opposite angles $v, w$.
        \begin{align*}
        	a\cos x+b\sin x&=r\cos(x-\phi)\\
        	a\sin x+b\cos x&=r\sin(x+\phi)
        \end{align*}
        where $r=\sqrt{a^2+b^2}, \phi=\operatorname{atan2}(b,a)$.
        

    \subsection{Combinatorics}
        Combinations and Permutations
        
        $P(n,r) = \frac{n!}{(n-r)!}$
        
        $C(n,r) = \binom{n}{r} = \frac{n!}{r!(n-r)!}$
        
        $C(n,r) = C(n, n-r)$

	\subsection{Cycles}
		Let $g_S(n)$ be the number of $n$-permutations whose cycle lengths all belong to the set $S$. Then
		$$\sum_{n=0} ^\infty g_S(n) \frac{x^n}{n!} = \exp\left(\sum_{n\in S} \frac{x^n} {n} \right)$$
		
	\subsection{Labeled unrooted trees}
		\# on $n$ vertices: $n^{n-2}$ \\
		\# on $k$ existing trees of size $n_i$: $n_1n_2\cdots n_k n^{k-2}$ \\
		\# with degrees $d_i$: $(n-2)! / ((d_1-1)! \cdots (d_n-1)!)$

	\subsection{Partition function}
		Number of ways of writing $n$ as a sum of positive integers, disregarding the order of the summands.
		\[ p(0) = 1,\ p(n) = \sum_{k \in \mathbb Z \setminus \{0\}}{(-1)^{k+1} p(n - k(3k-1) / 2)} \]
		\[ p(n) \sim 0.145 / n \cdot \exp(2.56 \sqrt{n}) \]

		\begin{center}
		\begin{tabular}{c|c@{\ }c@{\ }c@{\ }c@{\ }c@{\ }c@{\ }c@{\ }c@{\ }c@{\ }c@{\ }c@{\ }c@{\ }c}
			$n$    & 0 & 1 & 2 & 3 & 4 & 5 & 6  & 7  & 8  & 9  & 20  & 50  & 100 \\ \hline
			$p(n)$ & 1 & 1 & 2 & 3 & 5 & 7 & 11 & 15 & 22 & 30 & 627 & $\mathtt{\sim}$2e5 & $\mathtt{\sim}$2e8 \\
		\end{tabular}
		\end{center}
	
	\subsection{Numbers}
	\subsubsection*{Bernoulli numbers}
		EGF of Bernoulli numbers is $B(t)=\frac{t}{e^t-1}$ (FFT-able).
		$B[0,\ldots] = [1, -\frac{1}{2}, \frac{1}{6}, 0, -\frac{1}{30}, 0, \frac{1}{42}, \ldots]$

		Sums of powers:
		\small
		\[ \sum_{i=1}^n n^m = \frac{1}{m+1} \sum_{k=0}^m \binom{m+1}{k} B_k \cdot (n+1)^{m+1-k} \]
		\normalsize
		Euler-Maclaurin formula for infinite sums:
		\small
		\[ \sum_{i=m}^{\infty} f(i) = \int_m^\infty f(x) dx - \sum_{k=1}^\infty \frac{B_k}{k!}f^{(k-1)}(m) \]
		\[ \approx \int_{m}^\infty f(x)dx + \frac{f(m)}{2} - \frac{f'(m)}{12} + \frac{f'''(m)}{720} + O(f^{(5)}(m)) \]
		\normalsize

	\subsubsection*{Stirling's numbers} \textbf{First kind:} $S_1(n, k)$ count permutations on $n$ items
		with $k$ cycles. $S_1(n, k) = S_1(n-1, k-1) + (n-1)S_1(n-1, k)$ with
		$S_1(0, 0) = 1$. Note:
		$$\sum_{k=0}^n S_1(n, k)x^k = x(x+1)\dots(x+n-1)$$
		$$ \sum_{k=0}^n S_1(n, k) = n! $$
		$S_1(8,k) = 8, 0, 5040, 13068, 13132, 6769, 1960, 322, 28, 1$ \\
		$S_1(n,2) = 0, 0, 1, 3, 11, 50, 274, 1764, 13068, 109584, \dots$
		\textbf{Second kind:} $S_2(n, k)$ count partitions of $n$
		distinct elements into exactly $k$ non-empty groups.
		$$ S_2(n, k) = S_2(n-1, k-1) + kS_2(n-1, k)$$
		$$S_2(n, 1) = S_2(n, n) = 1$$
		$$ S_2(n, k) = \frac{1}{k!}\sum_{i=0}^k (-1)^{k-i}\binom{k}{i}i^n $$

	\subsubsection*{Catalan Numbers} - Number of correct bracket sequence consisting of $n$ opening and $n$
		closing brackets. \\
		The number of ways to completely parenthesize $n+1$ factors. \\
		The number of triangulations of a convex polygon with $n+2$
		sides (i.e. the number of partitions of polygon into disjoint triangles by using the diagonals). \\
		The number of ways to connect the $2n$
		points on a circle to form n disjoint i.e. non-intersecting chords.
		$$ C_n = \frac{1}{n+1} \binom{2n}{n}$$
		$$ C_0 = 1,~C_1 = 1, ~C_n = \sum_{k=0}^{n-1}C_k C_{n-1-k}  $$

	\subsubsection*{Narayana numbers} The number of expressions containing n pairs of parentheses, which are correctly matched and which contain k distinct nestings. 
		$$ N(n,k) = \frac{1}{n} \frac{n}{k} \frac{n}{k-1} $$

	\subsubsection*{Eulerian numbers}
		Number of permutations $\pi \in S_n$ in which exactly $k$ elements are greater than the previous element. $k$ $j$:s s.t. $\pi(j)>\pi(j+1)$, $k+1$ $j$:s s.t. $\pi(j)\geq j$, $k$ $j$:s s.t. $\pi(j)>j$.
		$$E(n,k) = (n-k)E(n-1,k-1) + (k+1)E(n-1,k)$$
		$$E(n,0) = E(n,n-1) = 1$$
		$$E(n,k) = \sum_{j=0}^k(-1)^j\binom{n+1}{j}(k+1-j)^n$$


	\subsubsection*{Bell numbers}
		Total number of partitions of $n$ distinct elements. $B(n) =$
		$1, 1, 2, 5, 15, 52, 203, 877, 4140, 21147, \dots$. For $p$ prime,
		\[ B(p^m+n)\equiv mB(n)+B(n+1) \pmod{p} \]

	\subsubsection*{Catalan numbers}
		\[ C_n=\frac{1}{n+1}\binom{2n}{n}= \binom{2n}{n}-\binom{2n}{n+1} = \frac{(2n)!}{(n+1)!n!} \]
		\[ C_0=1,\ C_{n+1} = \frac{2(2n+1)}{n+2}C_n,\ C_{n+1}=\sum C_iC_{n-i} \]
		${C_n = 1, 1, 2, 5, 14, 42, 132, 429, 1430, 4862, 16796, 58786, \dots}$
		\begin{itemize}[noitemsep]
			\item sub-diagonal monotone paths in an $n\times n$ grid.
			\item strings with $n$ pairs of parenthesis, correctly nested.
			\item binary trees with with $n+1$ leaves (0 or 2 children).
			\item ordered trees with $n+1$ vertices.
			\item ways a convex polygon with $n+2$ sides can be cut into triangles by connecting vertices with straight lines.
			\item permutations of $[n]$ with no 3-term increasing subseq.
		\end{itemize}

    \subsection{Probability} Stochastic variables

        $P(X=r) = C(n,r) \cdot p^r \cdot (1-p)^{n-r}$

        \subsubsection*{Bayes' Theorem}

        $P(B|A) = \frac{P(A|B) P(B)}{P(A)}$
        
        $P(B|A) = \frac{P(A|B)P(B)}{P(A|B)P(B) + P(A|\bar{B})P(\bar{B})}$
        
        $P(B_k|A) = \frac{P(A|B_k)P(B_k)}{P(A|B_1)P(B_1) \cdot ... \cdot P(A|B_n)P(B_n)}$

        \subsubsection*{Expectation}

        Let $X$ be a discrete random variable with probability $p_X(x)$ of assuming the value $x$. It will then have an expected value (mean) $\mu=\mathbb{E}(X)=\sum_xxp_X(x)$ and variance $\sigma^2=V(X)=\mathbb{E}(X^2)-(\mathbb{E}(X))^2=\sum_x(x-\mathbb{E}(X))^2p_X(x)$ where $\sigma$ is the standard deviation. If $X$ is instead continuous it will have a probability density function $f_X(x)$ and the sums above will instead be integrals with $p_X(x)$ replaced by $f_X(x)$.

	Expectation is linear:
	\[\mathbb{E}(aX+bY) = a\mathbb{E}(X)+b\mathbb{E}(Y)\]
	For independent $X$ and $Y$, \[V(aX+bY) = a^2V(X)+b^2V(Y).\]

    \subsection{Number Theory}
        \subsubsection*{Bezout's Theorem}
        $$
        a,b \in \mathbb{Z^+} \implies \exists s,t \in \mathbb{Z}: \, \gcd(a,b) = sa + tb
        $$
        \subsubsection*{Bézout's identity}
	For $a \neq $, $b \neq 0$, then $d=gcd(a,b)$ is the smallest positive integer for which there are integer solutions to
	$$ax+by=d$$
	If $(x,y)$ is one solution, then all solutions are given by
	$$\left(x+\frac{kb}{\gcd(a,b)}, y-\frac{ka}{\gcd(a,b)}\right), \quad k\in\mathbb{Z}$$

        \subsubsection*{Partial Coprime Divisor Property}
        $$
        (\gcd(a,b)=1) \land (a \mid bc) \implies (a \mid c)
        $$
        \subsubsection*{Coprime Modulus Equivalence Property}
        $$
        (\gcd(c,m)=1) \land (ac \equiv bc \mod m) \implies (a \equiv b \mod m)
        $$
        \subsubsection*{Fermat's Little Theorem}
        $$
        (\text{prime}(p)) \land (p \nmid a) \implies (a^{p-1} \equiv 1 \mod p)
        $$
        $$
        (\text{prime}(p)) \implies (a^p \equiv a \mod p) 
        $$
		\subsubsection*{Euler's Theorem}
		$$
		a^{\phi(m)-1} \equiv a^{-1} \mod m,~ \text{if}~ \gcd(a,m) = 1
		$$
		$$
		a^{-1} \equiv a^{m-2} \mod m,~ \text{if}~ m \text{ is prime}
		$$


\subsubsection*{Pythagorean Triples}
 The Pythagorean triples are uniquely generated by
 \[ a=k\cdot (m^{2}-n^{2}),\ \,b=k\cdot (2mn),\ \,c=k\cdot (m^{2}+n^{2}), \]
 with $m > n > 0$, $k > 0$, $m \bot n$, and either $m$ or $n$ even.

\subsubsection*{Primes}
	$p=962592769$ is such that $2^{21} \mid p-1$, which may be useful. For hashing
	use 970592641 (31-bit number), 31443539979727 (45-bit), 3006703054056749
	(52-bit). There are 78498 primes less than 1\,000\,000.

	Primitive roots exist modulo any prime power $p^a$, except for $p = 2, a > 2$, and there are $\phi(\phi(p^a))$ many.
	For $p = 2, a > 2$, the group $\mathbb Z_{2^a}^\times$ is instead isomorphic to $\mathbb Z_2 \times \mathbb Z_{2^{a-2}}$.

\subsubsection*{Estimates}
	$\sum_{d|n} d = O(n \log \log n)$.

	The number of divisors of $n$ is at most around 100 for $n < 5e4$, 500 for $n < 1e7$, 2000 for $n < 1e10$, 200\,000 for $n < 1e19$.

\subsubsection*{Mobius Function}
\[
	\mu(n) = \begin{cases} 0 & n \textrm{ is not square free}\\ 1 & n \textrm{ has even number of prime factors}\\ -1 & n \textrm{ has odd number of prime factors}\\\end{cases}
\]
  Mobius Inversion:
  \[ g(n) = \sum_{d|n} f(d) \Leftrightarrow f(n) = \sum_{d|n} \mu(d)g(n/d) \]
  Other useful formulas/forms:

  $ \sum_{d | n} \mu(d) = [ n = 1] $ (very useful)

  $ g(n) = \sum_{n|d} f(d) \Leftrightarrow f(n) = \sum_{n|d} \mu(d/n)g(d)$

 $ g(n) = \sum_{1 \leq m \leq n} f(\left\lfloor\frac{n}{m}\right \rfloor ) \Leftrightarrow f(n) = \sum_{1\leq m\leq n} \mu(m)g(\left\lfloor\frac{n}{m}\right\rfloor)$



\subsection{Discrete distributions}

\subsubsection*{Binomial distribution}
The number of successes in $n$ independent yes/no experiments, each which yields success with probability $p$ is $\textrm{Bin}(n,p),\,n=1,2,\dots,\, 0\leq p\leq1$.
\[p(k)=\binom{n}{k}p^k(1-p)^{n-k}\]
\[\mu = np,\,\sigma^2=np(1-p)\]
$\textrm{Bin}(n,p)$ is approximately $\textrm{Po}(np)$ for small $p$.

\subsubsection*{First success distribution}
The number of trials needed to get the first success in independent yes/no experiments, each wich yields success with probability $p$ is $\textrm{Fs}(p),\,0\leq p\leq1$.
\[p(k)=p(1-p)^{k-1},\,k=1,2,\dots\]
\[\mu = \frac1p,\,\sigma^2=\frac{1-p}{p^2}\]

\subsubsection*{Poisson distribution}
The number of events occurring in a fixed period of time $t$ if these events occur with a known average rate $\kappa$ and independently of the time since the last event is $\textrm{Po}(\lambda),\,\lambda=t\kappa$.
\[p(k)=e^{-\lambda}\frac{\lambda^k}{k!}, k=0,1,2,\dots\]
\[\mu=\lambda,\,\sigma^2=\lambda\]

\subsection{Continuous distributions}

\subsubsection*{Uniform distribution}
If the probability density function is constant between $a$ and $b$ and 0 elsewhere it is $\textrm{U}(a,b),\,a<b$.
\[f(x) = \left\{
\begin{array}{cl}
\frac{1}{b-a} & a<x<b\\
0 & \textrm{otherwise}
\end{array}\right.\]
\[\mu=\frac{a+b}{2},\,\sigma^2=\frac{(b-a)^2}{12}\]

\subsubsection*{Exponential distribution}
The time between events in a Poisson process is $\textrm{Exp}(\lambda),\,\lambda>0$.
\[f(x) = \left\{
\begin{array}{cl}
\lambda e^{-\lambda x} & x\geq0\\
0 & x<0
\end{array}\right.\]
\[\mu=\frac{1}{\lambda},\,\sigma^2=\frac{1}{\lambda^2}\]

\subsubsection*{Normal distribution}
Most real random values with mean $\mu$ and variance $\sigma^2$ are well described by $\mathcal{N}(\mu,\sigma^2),\,\sigma>0$.
\[ f(x) = \frac{1}{\sqrt{2\pi\sigma^2}}e^{-\frac{(x-\mu)^2}{2\sigma^2}} \]
If $X_1 \sim \mathcal{N}(\mu_1,\sigma_1^2)$ and $X_2 \sim \mathcal{N}(\mu_2,\sigma_2^2)$ then
\[ aX_1 + bX_2 + c \sim \mathcal{N}(\mu_1+\mu_2+c,a^2\sigma_1^2+b^2\sigma_2^2) \]

\end{multicols}
\end{document}
